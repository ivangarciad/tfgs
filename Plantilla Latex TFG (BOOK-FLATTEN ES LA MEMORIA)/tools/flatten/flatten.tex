% flatten.tex      Description of the flatten program
\ifx\documentclass\undefined
  \documentstyle[11pt]{article}
\else
  \documentclass[11pt]{article}
\fi
\setlength{\textheight}{8.0in}
\setlength{\textwidth}{6.0in}
\setlength{\oddsidemargin}{0.25in}
\setlength{\evensidemargin}{0.25in}
\setlength{\marginparwidth}{0.6in}
\setcounter{secnumdepth}{4}
\setcounter{tocdepth}{4}

% 2.09 definition of \LaTeX logo
%\def\p@LaTeX{{\reset@font\rm L\kern-.36em\raise.3ex\hbox{\sc a}\kern-.15em%
%              T\kern-.1667em\lower.7ex\hbox{E}\kern-.125emX}}
%\def\LaTeX{\protect\p@LaTeX}

\makeatletter
\def\p@fl{{FL\kern-.36em\raise.3ex\hbox{\sc a}\kern-.15em%
              TT\kern-.1667em\lower.7ex\hbox{E}\kern-.125emN}}
\def\fl{\protect\p@fl}
\makeatother

\newcommand{\sref}[1]{\S\ref{#1}}
\newcommand{\fid}[1]{{\tt #1}}
\newenvironment{code}{\begin{small}}{\end{small}}

\title{\fl: A Program to Flatten \LaTeX\ Source Files}
\author{Peter R. Wilson \\
        Catholic University of America and NIST \\
        Email: {\tt pwilson@cme.nist.gov} }
\date{October 1995}

\begin{document}
\bibliographystyle{unsrt}
\pagenumbering{roman}
\maketitle

\begin{abstract}
    \fl\ is a table-driven program that will automatically incorporate the
text of `included' files into a \LaTeX\ root file.
This report describes Version~1.1 of the program.
\end{abstract}

\tableofcontents

\clearpage
\pagenumbering{arabic}

\section{Introduction}


    \LaTeX\ is a popular document tagging and typesetting system~\cite{LAMPORT94}.
A \LaTeX\ source file can reference other \LaTeX\ files whose contents are to
be treated as an integral part of the source file. The standard \LaTeX\ has two
{\em inclusion} commands for this purpose, 
namely \verb|\input{FileName}| and \verb|\include{FileName}|.
In its turn, a file that is \verb|\input|ed can recursively \verb|\input| other
files. This \LaTeX\ facility provides a means of splitting a large document into
several files. Sometimes, though, it is desirable to combine the individual
files into a single source file. An example might be when storing a multi-file
\LaTeX\ document into a text database where documents are maintained as one file
per document.

    The \fl\ program will read a \LaTeX\ source file and automatically incorporate
the source of inclusion files that are designated via the appropriate \LaTeX\
commands.

\section{The program}

    \fl\ requires two files to be specified: the name of the input source \LaTeX\ file
and the name of the file to which the concatenated source is to be output. For
example, if \verb|>| is the system prompt, then
\begin{verbatim}
> flatten in.tex inall.tex
\end{verbatim}
will read the file \fid{in.tex} and write out the source to file \fid{inall.tex},
incorporating the source of any inclusion files identified within \fid{in.tex}. \fl\ will
write progress reports to the terminal, and also to an error file called
\fid{flatten.err}. This latter file can be examined using your favorite
text editor after a run if required.

    In the output file, \fl\ comments out any file inclusion commands, incorporates
the source from the inclusion file, adds a comment at the end of this source, and
then goes on outputting from the original source file. As far as \fl\
is concerned a file ends when either it reaches the end of the file or when
it comes across the \TeX\ \verb|\endinput| command before the file physically ends.

    If \fl\ comes across an inclusion file that it cannot find or read for any reason,
it prints a warning message, both to the user's terminal and as a
\LaTeX\ comment in the output file, and then
continues processing the current file.

    \fl\ correctly ignores any inclusion commands that are embedded within \LaTeX\
comments or are within verbatim text.

\subsection{Command file}

    By default, \fl\ uses the \LaTeX\ \verb|\input| and \verb|\include| commands
as the inclusion commands\footnote{\fl\ does not recognize the 
{\tt $\backslash$includeonly} command; do not attempt to make it do so.}. It may
happen that other file inclusion commands have been defined in a macro package
that is used by the source document. In this case, by using the \verb|-f| 
option (see~\sref{sec:running}),
\fl\ can be told at
runtime about the relevant inclusion commands to be used.

    The \verb|-f| option is used to specify a command table file. 
\fl\ reads the designated file which must contain
{\em all} the inclusion commands (including \verb|\input| and \verb|\include| if
relevant). It will then use these commands for deciding upon inclusions instead
of the default inclusions. The format
of the command file is very simple --- no blank lines and one inclusion command per
line. For example, here is a command file that contains three
inclusion commands:
\begin{code}
\begin{verbatim}
\input
          \include
    \infile
\end{verbatim}
\end{code}
Note that a command does not have to be at the start of a line.
In all cases, \fl\ assumes that an inclusion command, say \verb|\inclusion|, 
has the syntax
\begin{verbatim}
\inclusion{filename}
\end{verbatim}
and further assumes that if \fid{filename} has no extension, then it is
actually called \fid{filename.tex}. This means that \fl\ will not be able to
process an inclusion command like \verb|\usepackage| because 
(a) this command assumes a file name extension of \fid{.sty} and 
(b) the command can take an optional first parameter.

    An example \LaTeX\ source file could be like the following:
\begin{code}
\begin{verbatim}
% root.tex starts here
....
\infile{file1}
...
\input{file2.mac}
...
\include{fileN}
...
% root.tex ends here
\end{verbatim}
\end{code}

    Running \fl\ on this source file and the above command table file will 
produce an output file along the lines of:
\begin{code}
\begin{verbatim}
% root.tex starts here
...
%\infile{file1}
         % contents of file1(.tex) here
         % and any recursively called inclusions
%    END file1.tex
...
%\input{file2.mac}
         % contents of file2.mac here
         % and any recursively called inclusions
%    END file2.mac
...
%\include{fileN}
         % contents of fileN(.tex) here
         % and any recursively called inclusions
%    END fileN.tex
...
% root.tex ends here
\end{verbatim}
\end{code}
On the other hand, running \fl\ in its default mode will produce an output file
similar to:
\begin{code}
\begin{verbatim}
% root.tex starts here
...
\infile{file1}
...
%\input{file2.mac}
         % contents of file2.mac here
         % and any recursively called inclusions
%    END file2.mac
...
%\include{fileN}
         % contents of fileN(.tex) here
         % and any recursively called inclusions
%    END fileN.tex
...
% root.tex ends here
\end{verbatim}
\end{code}

\section{Running \fl\ } \label{sec:running}

    The full syntax of the command to run \fl\ is:
\begin{verbatim}
flatten [-d number] [-f CommandFile] [-D dir_cat_char] [-P path_separators]
        InFile OutFile
\end{verbatim}
where the square brackets enclose runtime options.
\begin{description}
\item[{\tt -d}] An option to print debugging information to \fid{flatten.err}.
A value of $1$ will print diagnostics related to file searching, while
a value of $2$ or more will produce full diagnostic information.

\item[{\tt -f}] An option to read file inclusion commands from the 
file \fid{CommandFile}.

\item[{\tt -D}] The value of this option is the character that the
operating system uses to catenate directory names to form a path.
The default value is a slash (i.e., \verb|/|). This could be changed
to a backslash, for example, by including the option \verb|-D \|.
(See~\sref{sec:srch}).

\item[{\tt -P}] The character(s) that the operating system uses to separate
path names in an environment variable. The default characters are space,
colon and semi-colon (i.e. \verb*| :;|). The separator characters could be changed
to colon and space by including the option \verb|-P :| (space
is always available as a separator character).
(See~\sref{sec:srch}).

\item[{\tt InFile}] The name of the input file (including any extension).

\item[{\tt OutFile}] The name of the output file.
\end{description}

\subsection{Directory searching} \label{sec:srch}

    The program employs a search algorithm to find files that are not
in the current directory. It first looks in the current directory and
if a file of the given name is found, then that is used. If the file
is not found, then it searches for it among directories that are
specified in a system environment variable. This variable specifies
a list of pathnames, where the directories forming the path are
combined using a catenation character. For example, \verb|dir1/dir2/dir3|
could be a pathname, where the slash (\verb|/|) is the catenation character.
If it is looking for file \verb|afile.txt| it will catenate the file name
to the path name (e.g. \verb|dir1/dir2/dir3/afile.txt|) and look for
that. The pathnames in the list are separated by another character
(in fact it can be one from a list of characters). For example here
is a list of two pathnames; \verb|dir1/dir2;dir1/dir4|, where the semi-colon
(\verb|;|) is the pathname separator.

    By default, the program uses a slash (\verb|/|) as the directory
catenation character and the pathname separators can be a space, or
a colon or a semi-colon (i.e., any of \verb*| :;|). All these characters
can be altered via the program command line options, and should be set to
match the conventions of your operating system.

    The environment variable used by the program is FLATINPUTS. 
On the operating system that I use, I set this in my login file like:
\begin{code}
\begin{verbatim}
setenv FLATINPUTS .:/dir1/dir2:/dir3/dir4
\end{verbatim}
\end{code}
Your system may have different conventions.
Note that if the environment variable is not set, then the
program only looks for files in the current directory.


\clearpage
\appendix

\section{System installation}

    This section describes how to install the \fl\ program.

    \fl\ is written using flex\footnote{flex is a version of the lex 
program~\protect\cite{LESK75} but with
increased functionality~\protect\cite{LEVINE92}.} 
and C~\cite{KERNIGHAN88}. The system consists of the following source files:
\begin{description}
\item[\fid{flatten.l}] The main source for \fl.
\item[\fid{getopt.c}] Functions for handling command line
                     arguments~\cite[Chapter 6]{LIBES93}.
\item[\fid{getopt.h}] A header file for \fid{getopt.c}.
\item[\fid{srchenv.c}] Functions for directory searching~\cite[page 747]{HOLUB90}.
\item[\fid{srchenv.h}] A header file for \fid{srchenv.c}
\item[\fid{man}] A manpage for \fl.
\end{description}

    The file \fid{flatten.l} has to be processed by flex, or an
equivalent tool, to generate C a file containing C code. This is
then compiled and linked, together with the other system files,
to form the executable program.
 
\subsection{A {\tt make} file}

    For those who understand such things, here is a \fid{make} 
file~\cite{ORAM91} for \fl.
\begin{code}
\begin{verbatim}
# makefile for program flatten --- include all "include" files
#                                  into a LaTeX root file
#
##################### Change the following for your setup
# The compiler
CC = cc

# We use flex (or equivalent) to generate the lexer
LEX = flex
# and the options
LEXFLAGS = -v

# Libraries to be used
LIBS = -ll -lm

# The root directory for installation (e.g., /usr/local )
ROOTDIR = /proj/ltx/teTeX033

# Where to place the running code (e.g. /usr/local/bin )
BINDIR = ${ROOTDIR}/bin

# Where to place the man page (e.g., /usr/local/man/man1 )
MANEXT = 1
MANDIR = ${ROOTDIR}/man/man${MANEXT}

# Just in case you want to change the name of the binary
# (and then you should also change the man page and documentation).
# So, do not change this.
PROG = flatten

# Where to place the user documentation (e.g., /usr/local/doc/flatten )
DOCDIR = ${ROOTDIR}/doc/${PROG}

# The file copy command (copy but do not delete original)
COPY = cp

# The file move command (move and delete original)
MOVE = mv

# The file delete command
DELETE = rm

# The make directory (hierarchy) command
MAKEDIR = mkdirhier

# The stream editor command
SED = sed

################### You should not have to change anything after this

# The object modules
OBJS = flatten.o getopt.o srchenv.o

# Link object code together into PROG
flatten : ${OBJS}
        ${CC} -o ${PROG} ${OBJS} ${LIBS}

# Compile C source code into object code
flatten.o : flatten.c getopt.h srchenv.h
        ${CC} -c flatten.c
getopt.o : getopt.c getopt.h
        ${CC} -c getopt.c
srchenv.o : srchenv.c
        ${CC} -c srchenv.c

# Generate C code via LEXing
flatten.c : flatten.l
        ${LEX} ${LEXFLAGS} flatten.l
        ${MOVE} lex.yy.c flatten.c

# Only call make install if BINDIR has been set
install : flatten
        ${MAKEDIR} ${BINDIR}
        ${MOVE} ${PROG} ${BINDIR}

# Edit file man to replace DOCUMENTDIR by the actual directory
# where the user manual is to be placed.
# Then copy the man page to the proper place.
manpage :
        ${SED} 's!DOCUMENTDIR!${DOCDIR}!' man > tman
        ${MAKEDIR} ${MANDIR}
        ${COPY} tman ${MANDIR}/${PROG}.${MANEXT}

# Copy the user manuals to the proper place
doc :
        ${MAKEDIR} ${DOCDIR}
        ${COPY} flatten.tex ${DOCDIR}/${PROG}.tex
        ${COPY} flatten.ps ${DOCDIR}/${PROG}.ps

# Do everything except clean up
all : flatten install manpage doc

# Call make clean to remove object files and edited man page
clean :
        ${DELETE} *.o
        ${DELETE} tman
\end{verbatim}
\end{code}

    When using this \fid{makefile}, edit the first part of the file
to configure it for your system and environment. If you are brave,
you can then do \verb|make all| followed by \verb|make clean|.
Otherwise just do \verb|make| which will compile and link \fl.
After testing \fl, do \verb|make install| to move the binary
to its proper location, \verb|make manpage| to edit and copy the
manpage to its desired location, and \verb|make doc| to copy
the user documentation to its final location. Finally, do 
\verb|make clean| to get rid of the temporary files.


\subsection{Limits}

    There are some built-in limits within \fl. If necessary, these can be changed and the
system re-built.

\begin{description}
\item[{\tt MAX\_COMMANDS}] The maximum number of inclusion commands within a
   command file.
\item[{\tt MAX\_DEPTH}] The maximum allowable depth of file nesting. This is set to
   a value that is probably greater than the number of simultaneously open files 
  that your operating (and \LaTeX) system can support.
\item[{\tt MAX\_ERRORS}] The maximum number of errors before \fl\ gives up.
\item[{\tt MAX\_NAME}] The maximum number of characters in a file name.
\item[{\tt MAX\_LINE}] The maximum length of a line in a \LaTeX\ input file.
\item[{\tt MAX\_TABLE\_LINE}] The maximum number of characters for an inclusion command
   in the command file.
\end{description}

\subsection{Availability}

    Source code and documentation for \fl\ is available from the
NIST SOLIS (STEP On-Line Information Service) system~\cite{RINAUDOT94}
in directory
{\tt pub/step/latex/programs/flatten}. SOLIS can be accessed by:
\begin{itemize}
\item Anonymous ftp to {\tt ftp.cme.nist.gov}
\item URL {\tt gopher://elib.cme.nist.gov}
\item URL {\tt http://elib.cme.nist.gov:70/}
\end{itemize}

\subsubsection{Copyright}

    Development of this software was funded by the United States
Government and is not subject to copyright.\\
National Institute of Standards and Technology (NIST) \\
Manufacturing Engineering Laboratory (MEL) \\
Manufacturing Systems Integration Division (MSID)

\subsubsection{Disclaimer}

    There is no warranty for the \fl\ software. If the \fl\ software
is modified by someone else and passed on, NIST wants the software's
recipients to know that what they have is not what NIST distributed.

\begin{description}
\item[Policies] \mbox{}
\begin{enumerate}
\item Anyone may copy and distribute verbatim copies of the
      source code as received in any medium.
\item Anyone may modify your copy or copies of the \fl\ source
      code or any portion of it, and copy and distribute such
      modifications provided that all modifications are clearly 
      associated with the entity that performs the modifications.
\end{enumerate}

\item[NO WARRANTY] \mbox{}

    NIST PROVIDES ABSOLUTELY NO WARRANTY. THE \fl\ SOFTWARE IS
PROVIDED `AS IS' WITHOUT WARRANTY OF ANY KIND, EITHER EXPRESSED
OR IMPLIED, INCLUDING, BUT NOT LIMITED TO, THE IMPLIED WARRANTIES
OF MERCHANTABILITY AND FITNESS FOR A PARTICULAR PURPOSE. THE ENTIRE
RISK AS TO THE QUALITY AND PERFORMANCE OF THE PROGRAM IS WITH YOU.
SHOULD ANY PORTION OF THE \fl\ SOFTWARE PROVE DEFECTIVE, YOU ASSUME
THE COST OF ALL NECESSARY SERVICING, REPAIR OR CORRECTION.

    IN NO EVENT WILL NIST BE LIABLE FOR DAMAGES, INCLUDING ANY
LOST PROFITS, LOST MONIES, OR OTHER SPECIAL, INCIDENTAL OR
CONSEQUENTIAL DAMAGES ARISING OUT OF THE USE OR INABILITY TO USE
(INCLUDING BUT NOT LIMITED TO LOSS OF DATA OR DATA BEING RENDERED
INACCURATE OR LOSSES SUSTAINED BY THIRD PARTIES OR A FAILURE OF
THE PROGRAM TO OPERATE WITH PROGRAMS NOT DISTRIBUTED BY NIST)
THE PROGRAMS, EVEN IF YOU HAVE BEEN ADVISED OF THE POSSIBILITY 
OF SUCH DAMAGES, OR FOR ANY CLAIM BY ANY OTHER PARTY.
\end{description}

\addcontentsline{toc}{section}{References}
%\bibliography{/auto/home/pwilson/Rpi/refs,/auto/home/pwilson/Rpi/prw}

\begin{thebibliography}{1}

\bibitem{LAMPORT94}
Leslie Lamport.
\newblock {\em LaTeX: A Document Preparation System}.
\newblock Addison-Wesley Publishing Company, second edition, 1994.

\bibitem{LESK75}
M.~E. Lesk and E.~Schmidt.
\newblock `{LEX --- A Lexical Analyser Generator}'.
\newblock In {\em UNIX Programmer's Manual 2}. AT\&T Bell Laboratories, Murray
  Hill, NJ, 1975.

\bibitem{LEVINE92}
John~R. Levine, Tony Mason, and Doug Brown.
\newblock {\em lex \& yacc}.
\newblock O'Reilly \& Associates, Inc., second edition, 1992.

\bibitem{KERNIGHAN88}
Brian~W. Kernighan and Dennis~M. Ritchie.
\newblock {\em The C Programming Language}.
\newblock Prentice Hall, second edition, 1988.

\bibitem{LIBES93}
Don Libes.
\newblock {\em Obfuscated C and Other Mysteries}.
\newblock John Wiley \& Sons, Inc., 1993.

\bibitem{HOLUB90}
A.~I. Holub.
\newblock {\em Compiler Design in C}.
\newblock Prentice-Hall, Inc., 1990.

\bibitem{ORAM91}
Andrew Oram and Steve Talbott.
\newblock {\em Managing Projects with make}.
\newblock O'Reilly \& Associates, Inc., second edition, 1991.

\bibitem{RINAUDOT94}
Gaylen~R. Rinaudot.
\newblock {\em {The IGES/PDES Organization STEP On-Line Information Service}}.
\newblock NISTIR 5511, NIST, Gaithersburg, MD 20899, October 1994.

\end{thebibliography}


\end{document}

